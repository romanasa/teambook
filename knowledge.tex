\texttt{
dbl Simpson() { return (F(-1) + 4 * F(0) + F(1)) / 6; }
dbl Runge2() { return (F(-sqrtl(1.0 / 3)) + F(sqrtl(1.0 / 3))) / 2; }
dbl Runge3() { return (F(-sqrtl(3.0 / 5)) * 5 + F(0) * 8 + F(sqrtl(3.0 / 5)) * 5) / 18; }
}
Simpson и Runge2 -- точны для полиномов степени <= 3
Runge3 -- точен для полиномов степени <= 5

---

Явный Рунге-Кутт четвертого порядка, ошибка $\mathcal{O}(h^4)$

$y' = f(x, y)$

$y_{n+1} = y_n + (k1 + 2 \cdot k2 + 2 \cdot k3 + k4) \cdot h / 6$

$k1 = f(xn, yn)$

$k2 = f(xn + h/2, yn + h/2 \cdot k1)$

$k3 = f(xn + h/2, yn + h/2 \cdot k2)$

$k4 = f(xn + h, yn + h \cdot k3)$

Методы Адамса-Башфорта

$y_{n+3} = y_{n+2} + h * (23/12 * f(x_{n+2},y_{n+2}) - 4/3 * f(x_{n+1},y_{n+1}) + 5/12 * f(x_n,y_n))$

$y_{n+4} = y_{n+3} + h * (55/24 * f(x_{n+3},y_{n+3}) - 59/24 * f(x_{n+2},y_{n+2}) + 37/24 * f(x_{n+1},y_{n+1}) - 3/8 * f(x_n,y_n))$

$y_{n+5} = y_{n+4} + h * (1901/720 * f(x_n+4,y_n+4) - 1387/360 * f(x_{n+3},y_{n+3})
    + 109/30 * f(x_{n+2},y_{n+2}) - 637/360 * f(x_{n+1},y_{n+1}) + 251/720 * f(x_n,y_n))$

---

Извлечение корня по простому модулю (от Сережи)
$3 \leq p$, $1 \leq a < p$, найти $x^2 = a$

1) Если $a^{\frac{p - 1}{2}} \ne 1$, return $-1$
2) Выбрать случайный $1 \leq i < p$
3) $T(x) = (x + i)^{(p - 1)/2} mod (x^2 - a) = bx + c$
4) Если $b \ne 0$ то вернуть $\frac{c}{b}$, иначе к шагу 2)

---

Иногда вместо того чтобы считать первообразный у простого числа,
можно написать чекер ответа и перебирать случайный первообразный.

Иногда можно представить ответ в виде многочлена и вместо подсчета самих к-тов посчитать значения и проинтерполировать

---
Чтобы посчитать количество остовных деревьев в неориентированном графе $G$:

    создать матрицу $N \times N$ \texttt{mat}, для каждого ребра ($a, b$):
    
	\texttt{mat[a][a]++, mat[b][b]++, mat[a][b]--, mat[b][a]--}.
	
	Удалить последнюю строку и столбец, взять дискриминант.
---

Лемма Бернсайда:

Группа G действует на множество X
Тогда число классов эквивалентности = (sum |f(g)| for g in G) / |G|
где f(g) = число x (из X) : g(x) == x

---

Число простых быстрее O(n): 
\texttt{
dp(n, k) -- число чисел от 1 до n в которых все простые >= p[k]
dp(n, 1) = n
dp(n, j) = dp(n, j + 1) + dp(n / p[j], j), т. е. 
dp(n, j + 1) = dp(n, j) - dp(n / p[j], j)
}
Если p[j], p[k] > sqrt(n) то dp(n, j) + j == dp(n, k) + k

Делаешь все оптимайзы сверху, но не считаешь глубже dp(n, k), n < K
Потом фенвиком+сортировкой подсчитываешь за (K+Q)log все эти запросы
Делаешь во второй раз, но на этот раз берешь прекальканные значения

Если sqrt(n) < p[k] < n то (число простых до n)=dp(n, k) + k - 1

---

$sum(k=1..n) k^2 = n(n+1)(2n+1)/6$

$sum(k=1..n) k^3 = n^2(n+1)^2/4$


Чиселки: 

Фибоначчи
45:  1134903170
46:  1836311903
47:  2971215073
91:  4660046610375530309
92:  7540113804746346429
93:  12200160415121876738

Числа с кучей делителей
20: d(12)=6
50: d(48)=10
100: d(60)=12
1000: d(840)=32
$10^4$: d(9240)=64
$10^5$: d(83160)=128
$10^6$: d(720720)=240
$10^7$: d(8648640)=448
$10^8$: d(91891800)=768
$10^9$: d(931170240)=1344
$10^{11}$: d(97772875200)=4032
$10^{12}$: d(963761198400)=6720
$10^{15}$: d(866421317361600)=26880
$10^{18}$: d(897612484786617600)=103680

Bell numbers:
$B(p^m + n) = mB(n) + B(n + 1) (mod p)$

0:1, 1:1, 2:2, 3:5, 4:15, 5:52, 6:203, 7:877, 8:4140, 9:21147,
10:115975, 11:678570, 12:4213597, 13:27644437, 14:190899322,
15:1382958545, 16:10480142147, 17:82864869804, 18:682076806159,
19:5832742205057, 20:51724158235372, 21:474869816156751,
22:4506715738447323, 23:44152005855084346

Catalan numbers:
$C_n = \binom{2n}{n} / (n + 1) = \binom{2n + 1}{n} / (2n + 1) = \binom{2n}{n} - \binom{2n}{n - 1}$

0:1, 1:1, 2:2, 3:5, 4:14, 5:42, 6:132, 7:429, 8:1430, 9:4862,
10:16796, 11:58786, 12:208012, 13:742900, 14:2674440,
15:9694845, 16:35357670, 17:129644790, 18:477638700,
19:1767263190, 20:6564120420, 21:24466267020, 22:91482563640,
23:343059613650, 24:1289904147324, 25:4861946401452

Partitions numbers:
see partition.cpp

0:1, 1:1, 2:2, 3:3, 4:5, 5:7, 6:11, 7:15, 8:22, 9:30, 10:42, 20:627, 30:5604, 40:37338, 50:204226, 60:966467, 70:4087968, 80:15796476, 90:56634173, 100:190569292

$prod (k=1..+inf) (1-x^k) = sum(q=-inf..+inf) (-1)^q x^((3q^2-q)/2)$